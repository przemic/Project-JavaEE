\documentclass[a4paper]{article}
\usepackage[margin=2cm]{geometry}

\usepackage{multirow}
 

\usepackage{polski}
\usepackage[utf8]{inputenc}
\usepackage[polish]{babel}

\usepackage{color, colortbl}
\usepackage[dvipsnames]{xcolor}
\usepackage{listings}
\usepackage{graphicx}

\usepackage{setspace}

\usepackage{natbib}

\lstset{literate=%
{ą}{{\k{a}}}1
{ć}{{\'c}}1
{ę}{{\k{e}}}1
{ł}{{\l{}}}1
{ń}{{\'n}}1
{ó}{{\'o}}1
{ś}{{\'s}}1
{ż}{{\.z}}1
{ź}{{\'z}}1
{Ą}{{\k{A}}}1
{Ć}{{\'C}}1
{Ę}{{\k{E}}}1
{Ł}{{\L{}}}1
{Ń}{{\'N}}1
{Ó}{{\'O}}1
{Ś}{{\'S}}1
{Ż}{{\.Z}}1
{Ź}{{\'Z}}1,
language=c++,
keywordstyle=\color{gray},
commentstyle=\color{lightgray},
stringstyle=\color{lightgray},
numbers=left,
frame=TBlr,
breaklines=true,
tabsize=2,
numberstyle=\tiny,
basicstyle=\footnotesize \ttfamily
}

\renewcommand\thesection{\arabic{section}.}
\renewcommand\thesubsection{\thesection\arabic{subsection}.}
\renewcommand\thesubsubsection{\thesubsection\arabic{subsubsection}.}
\renewcommand\theparagraph{\thesubsubsection\arabic{paragraph}.}
\renewcommand\thesubparagraph{\theparagraph\arabic{subparagraph}.}
\onehalfspacing
% Title Page
\title{Dokumentacja projektu  - inżynieria E-systemów w technologii JAVA \\ Aplikacja zarządzająca wydarzeniami kulturalnymi}
\author{\textbf{Przemysław Michalak} 181101\\ 
\textbf{Krystian Horecki} 181079 \\
\textbf{Grupa C} (śr. 11.15) \\ \\ Politechnika Wrocławska}
\date{20.05.2012 r.}

\begin{document}

\maketitle
\tableofcontents

\newpage


\section{Wymagania projektu}

Wymagania aplikacji zostały nieznacznie zmienione w miarę powstawania projektu.
Przy odpowiednich zmienionych wymaganiach zostały zamieszczone komentarze odnośnie zmienionych kryteriów oraz powodów takich zmian.

\subsection{Aplikacja internetowa}

\begin{table}[h!] 
\centering
\caption{Wymaganie funkcjonalne aplikacji internetowej FUN\_INT1}

\begin{tabular}{|p{2cm}|p{12cm}|} 

\hline	
	\multicolumn{2}{|>{\columncolor[gray]{.8}}c|}{Znajdowanie najbliższych wydarzeń	w sensie geograficznym}\\ \hline ID & FUN\_INT1 \\ 
	\hline \hline \multirow{2}{*}{Opis} & Użytkownik ma możliwość znalezienia wydarzeń  
	  znajdujących się najbliżej miejsca położenia podanego w aplikacji   \\	 
	\hline
	Priorytet & Wymagane \\ \hline	
	
\end{tabular}
\label{fun_int1}
\end{table}

Z uwagi na zmianę charakteru aplikacji, z nastawiania na wszelkie możliwe miejsca, na lokalizacje związane głównie z Wrocławiem, zrezygnowaliśmy z tej funkcjonalności.
Nie da się ukryć, iż implementacja posiadałaby duży nakład niepotrzebnej pracy.


\begin{table}[h!] 
\centering
\caption{Wymaganie funkcjonalne aplikacji internetowej FUN\_INT2}
\begin{tabular}{|p{2cm}|p{12cm}|} \hline	
	\multicolumn{2}{|>{\columncolor[gray]{.8}}c|}{Użytkownik ma możliwość dodawania wydarzeń}\\ \hline ID & FUN\_INT2 \\ \hline \hline
	 \multirow{2}{*}{Opis} &  Możliwe jest dodawanie przez użytkowników wydarzeń,   
	 które muszą być zatwierdzane przez administratora by mogły się pojawić 
	 w aplikacji, widoczne dla użytkowników \\
	 \hline Priorytet & Wymagane \\ \hline	
	
\end{tabular}
\label{fun_int2}
\end{table}

\begin{table}[h!] 
\centering
\caption{Wymaganie funkcjonalne aplikacji internetowej FUN\_INT3}
\begin{tabular}{|p{2cm}|p{12cm}|} \hline	
	\multicolumn{2}{|>{\columncolor[gray]{.8}}c|}{Przeglądanie wydarzeń}\\ \hline
	 ID & FUN\_INT3 \\ \hline \hline
	 \multirow{2}{*}{Opis} &  Aplikacja umożliwia przeglądanie wydarzeń według
	 wybranego kryterium (czas, miejsce, kategoria) \\  \hline 
	 Priorytet & Wymagane \\
	 \hline
	
\end{tabular}
\label{fun_int3}
\end{table}

W wypadku tej funkcjonalności uproszczony został system przeglądania wydarzeń, możliwe jest przeglądanie ich zgodnie z chronologią 
ich daty odbywania się, ma to związek z faktem braku odpowiedniego pola w bazie danych takiego jak kategoria wydarzenia, a także 
z uwagami zamieszczonymi odnośnie wymaganiem pierwszym.

\begin{table}[h!] 
\centering
\caption{Wymaganie funkcjonalne aplikacji internetowej FUN\_INT4}
\begin{tabular}{|p{2cm}|p{12cm}|} \hline	
	\multicolumn{2}{|>{\columncolor[gray]{.8}}c|}{Użytkownik ma możliwość podglądu miejsca wydarzenia}\\
	\hline ID & FUN\_INT4 \\ \hline \hline
	 \multirow{2}{*}{Opis} &  Aplikacja umożliwia uzyskanie podglądu na mapie
	 miejsca, w którym będzie odbywać się wydarzenie \\
	 \hline Priorytet & Wymagane \\
	 \hline
	
\end{tabular}
\label{fun_int4}
\end{table}

\begin{table}[h!] 
\centering
\caption{Wymaganie funkcjonalne aplikacji internetowej FUN\_INT5}
\begin{tabular}{|p{2cm}|p{12cm}|} \hline	
	\multicolumn{2}{|>{\columncolor[gray]{.8}}c|}{Aplikacja umożliwia załorzenie konta dla użytkownika}\\
	\hline ID & FUN\_INT5 \\ \hline \hline
	 \multirow{2}{*}{Opis} &  Możliwe jest założenie konta przez użytkownika,
	  oraz jego aktywacja poprzez email. \\
	  \hline Priorytet & Wymagane \\ \hline
	
\end{tabular}
\label{fun_int5}
\end{table}

W celu ułatwienia dostępu dla użytkowników walidacja konta odbywa się poprzez captche, a nie jak początkowo zakładano poprzez email.


\begin{table}[h!] 
\centering
\caption{Wymaganie funkcjonalne aplikacji internetowej FUN\_INT6}
\begin{tabular}{|p{2cm}|p{12cm}|} \hline	
	\multicolumn{2}{|>{\columncolor[gray]{.8}}c|}{Aplikacja umożliwia użytkownikowi dodawanie wydaerzeń do obserwowanych }\\ 
	\hline ID & FUN\_INT6 \\ \hline
	 \hline \multirow{2}{*}{Opis} &  Możliwe jest dodawanie przez użytkownika 
	 wydarzeń do ulubionych, a następnie ich podgląd na stronie konta użytkownika.
	 \\ \hline Priorytet & Wymagane
	 \\
	 \hline
	
\end{tabular}
\label{fun_int6}
\end{table}
\pagebreak
\subsection{Aplikacja mobilna}

\begin{table}[h!] 
\centering
\caption{Wymaganie funkcjonalne aplikacji mobilnej FUN\_MOB1}
\begin{tabular}{|p{2cm}|p{12cm}|} \hline	
	\multicolumn{2}{|>{\columncolor[gray]{.8}}c|}{Aplikacja umożliwia wyświetlenie wydarzeń }\\
	\hline ID & FUN\_MOB1 \\ \hline
	 \hline \multirow{2}{*}{Opis} &  Aplikacja umożliwia użytkownikowi wyświetlenie
	 wydarzeń na ekranie telefonu komórkowego. 
	 Możliwy jest określenie zakresu wyświetlania według kategorii, miejsca i
	 czasu. \\ 
	 \hline Priorytet & Wymagane
	 \\
	 \hline
	
\end{tabular}
\label{fun_mob1}
\end{table}

Uwagi identyczne jak odnośnie analogicznego wymagania dla aplikacji internetowej.

\begin{table}[h!] 
\centering
\caption{Wymaganie funkcjonalne aplikacji mobilnej FUN\_MOB2}
\begin{tabular}{|p{2cm}|p{12cm}|} \hline	
	\multicolumn{2}{|>{\columncolor[gray]{.8}}c|}{Aplikacja umożliwia zalogowanie się użytkownika }\\
	\hline ID & FUN\_MOB2 \\ \hline
	 \hline \multirow{2}{*}{Opis} &  Użytkownik ma możliwość zalogowania się do
	 aplikacji, co daje mu więcej funkcjonalności (dodawanie wydarzeń do
	 ulubionych). \\ \hline Priorytet & Wymagane
	 \\
	 \hline
	
\end{tabular}
\label{fun_mob2}
\end{table}

\begin{table}[h!] 
\centering
\caption{Wymaganie funkcjonalne aplikacji mobilnej FUN\_MOB3}
\begin{tabular}{|p{2cm}|p{12cm}|} \hline	
	\multicolumn{2}{|>{\columncolor[gray]{.8}}c|}{Aplikacja umożliwia dodawanie	wydarzeń do ulubionych }\\ 
	\hline ID & FUN\_MOB3 \\ \hline
	 \hline \multirow{2}{*}{Opis} & Użytkownik po zalogowaniu się, ma
	 możliwość dodawania wydarzeń do ulubionych. \\
	 \hline Priorytet & Wymagane
	 \\
	 \hline
	
\end{tabular}
\label{fun_mob3}
\end{table}


\begin{table}[h!] 
\centering
\caption{Wymaganie funkcjonalne aplikacji mobilnej FUN\_MOB4}
\begin{tabular}{|p{2cm}|p{12cm}|} \hline	
	\multicolumn{2}{|>{\columncolor[gray]{.8}}c|}{Aplikacja umożliwia poprowadzenie	użytkownika do miejsca wydarzenia}\\
	 \hline ID & FUN\_MOB4 \\ 
	\hline \hline
	 \multirow{2}{*}{Opis} &  Użytkownik po wybraniu wydarzenia ma możliwość
	 uzyskania widoku mapy, wraz z podaną trasą od miejsca aktualnego położenia
	 do miejsca w którym odbywa się wydarzenie.\\ 
	 \hline Priorytet & Wymagane
	 \\
	 \hline
	
\end{tabular}
\label{fun_mob4}
\end{table}
\pagebreak
\section{Wymagania niefunkcjonalne}
\subsection{Aplikacja internetowa}
\begin{table}[h!] 
\centering
\caption{Wymaganie niefunkcjonalne aplikacji internetowej NFUN\_INT1}
\begin{tabular}{|p{2cm}|p{12cm}|} \hline	
	\multicolumn{2}{|>{\columncolor[gray]{.8}}c|}{Aplikacja posiada przejrzysty	interfejs graficzny}\\ 
	\hline ID & NFUN\_INT1 \\ 
	\hline \hline
	 \multirow{2}{*}{Opis} &  Aplikacja dostarcza przejrzystego i prostego
	 interfejsu graficznego, o minimalistycznym charakterze \\
	 \hline
	 Priorytet & Wymagane
	 \\
	 \hline
	
\end{tabular}
\label{nfun_int1}
\end{table}

\begin{table}[h!] 
\centering
\caption{Wymaganie niefunkcjonalne aplikacji internetowej NFUN\_INT2}
\begin{tabular}{|p{2cm}|p{12cm}|} \hline	
	\multicolumn{2}{|>{\columncolor[gray]{.8}}c|}{Aplikacja posiada możliwość pobierania danych z innych serwisów}\\ 
	\hline ID & NFUN\_INT2 \\ 
	\hline \hline
	 \multirow{3}{*}{Opis} &  Aplikacja posiada możliwość pobierania danych oraz parsowania ich, w celu uzyskiwania 
	 informacji o wydarzeniach. Dane pobierane będą z grup i wydarzeń na stronie Facebook oraz strony
	 CouchSurfing. \\
	 \hline
	 Priorytet & Wymagane
	 \\
	 \hline
	
\end{tabular}
\label{nfun_int2}
\end{table}
W przypadku tego wymagania, zastosowano pobieranie wiadomości ze strony poświęconej Wrocławiu zamiast stron podanych w wymaganiach.
Dodatkowo, jako że parsowanie stron takich jak Facebook lub CouchSurfing stanowi duży problem, funkcjonalności została zrealizowana przy pomocy czytnika RSS.

\pagebreak
\section{Wymagania techniczne}
\subsection{Aplikacja internetowa}
\begin{table}[h!] 
\centering
\caption{Wymaganie techniczne aplikacji internetowej TECH\_INT1}
\begin{tabular}{|p{2cm}|p{12cm}|} \hline	
	\multicolumn{2}{|>{\columncolor[gray]{.8}}c|}{Język programowania}\\ 
	\hline ID & TECH\_INT1 \\ 
	\hline \hline
	 \multirow{2}{*}{Opis} & Aplikacja zostanie stworzona przy wykorzystaniu języka
	 Java EE w wersji 6 lub nowszej. \\
	 \hline
	
\end{tabular}
\label{tech_int1}
\end{table}

\begin{table}[h!] 
\centering
\caption{Wymaganie techniczne aplikacji internetowej TECH\_INT2}
\begin{tabular}{|p{2cm}|p{12cm}|} \hline	
	\multicolumn{2}{|>{\columncolor[gray]{.8}}c|}{System operacyjny}\\ 
	\hline ID & TECH\_INT2 \\ 
	\hline \hline
	 \multirow{1}{*}{Opis} & Linux \\
	 \hline
	
\end{tabular}
\label{tech_int2}
\end{table}


\subsection{Aplikacja mobilna}

\begin{table}[h!] 
\centering
\caption{Wymaganie techniczne aplikacji internetowej TECH\_MOB1}
\begin{tabular}{|p{2cm}|p{12cm}|} \hline	
	\multicolumn{2}{|>{\columncolor[gray]{.8}}c|}{Język programowania}\\ 
	\hline ID & TECH\_MOB1 \\ 
	\hline \hline
	 \multirow{2}{*}{Opis} & Aplikacja zostanie stworzona przy wykorzystaniu języka
	 Java oraz frameworka do tworzenia aplikacji na systemy Android. \\
	 \hline
	
\end{tabular}
\label{tech_mob1}
\end{table}


\begin{table}[h!] 
\centering
\caption{Wymaganie techniczne aplikacji internetowej TECH\_MOB2}
\begin{tabular}{|p{2cm}|p{12cm}|} \hline 	
	\multicolumn{2}{|>{\columncolor[gray]{.8}}c|}{Platforma}\\ 
	\hline ID & TECH\_MOB2 \\ 
	\hline \hline
	 \multirow{2}{*}{Opis} & Aplikacja zostanie stworzona na telefony posiadające system Android w wersji 2.1 lub nowszej. \\
	 \hline
	
\end{tabular}
\label{tech_mob2}
\end{table}

\section{Schemat bazy danych}
Poniżej na rysunku \ref{baza} zamieszczony został schemat bazy danych, zawierającej wszystkie wymagane do działania aplikacji
elementy.
Każde pole w bazie zostało na nim wstępnie opisane.
Na podstawie poniższego schematu w dużej mierze oparta zostanie późniejsza aplikacja.

\begin{figure}[!h]
\begin{center}
  \includegraphics[width=\textwidth]{baza_mod.png}
  \caption{Schemat bazy danych dla aplikacji}
  \label{baza}
\end{center}
\end{figure}

\pagebreak
\section{Diagram klas aplikacji}
Poniższy diagram przedstawiony na rysunku \ref{klasy} pokazuje uproszczoną, abstrakcyjną strukturę aplikacji.
Przedstawione na nim zależności pokazują strukturę w jakiej zorganizowany został przepływ danych w naszej aplikacji.
Pozwala on na oddzielenie funkcjonalności związanej z bezpośrednią obsługą prostych zapytań do bazy danych, od bardziej skomplikowanych
operacji biznesowych.


\begin{figure}[!h]
\begin{center}
  \includegraphics[width=\textwidth]{classDiagram.png}
  \caption{Diagram klas dla aplikacji}
  \label{klasy}
\end{center}
\end{figure}


\section{Dokumentacja aplikacji internetowej}

\subsection{Wykorzystane środowisko programistyczne}

\subsection{Wykorzystane technologie}

\subsection{Wykorzystywane biblioteki oraz frameworki}

\subsection{Funkcjonalności aplikacji}
\subsubsection{Z punktu widzenia zwykłego użytkownika}

\subsubsection{Z punktu widzenia administratora}

\subsection{Widoki aplikacji}

\section{Dokumentacja aplikacji mobilnej}

\subsection{Wykorzystane środowisko programistyczne}

\subsection{Wykorzystane technologie}

\subsection{Wykorzystywane biblioteki oraz frameworki}

\subsection{Funkcjonalności aplikacji}

\subsection{Widoki aplikacji}

\begin{lstlisting}[caption={Kod pliku Screen.c}, label={sc}]
 
\end{lstlisting}
\end{document} 




